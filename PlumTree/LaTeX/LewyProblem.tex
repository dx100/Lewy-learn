\documentclass[11pt, oneside]{article}   	% use "amsart" instead of "article" for AMSLaTeX format
\usepackage{geometry}                		% See geometry.pdf to learn the layout options. There are lots.
\geometry{letterpaper}                   		% ... or a4paper or a5paper or ... 
%\geometry{landscape}                		% Activate for rotated page geometry
%\usepackage[parfill]{parskip}    		% Activate to begin paragraphs with an empty line rather than an indent
\usepackage{graphicx}				% Use pdf, png, jpg, or eps§ with pdflatex; use eps in DVI mode
								% TeX will automatically convert eps --> pdf in pdflatex		
\usepackage{amssymb}

%SetFonts

%SetFonts
\renewcommand{\arraystretch}{1.5}

\title{Lewy's Plum Tree Problem}
\author{Lewy and Da}
%\date{}							% Activate to display a given date or no date

\begin{document}
\maketitle
%\section{}
%\subsection{}

\section{The Problem}
There are 100 fresh plums on a plum tree including 3 rotten ones. Every day, each plum has a 35\% chance of rotting, if they are over 2 days old. Every day, 3 fresh plums grow. The farmer comes and takes 75\% of the rotten plums every 4 days, stopping the rot by 2 days. After 2 weeks:

\begin{itemize}
\item How many fresh plums are there?
\item How many rotten plums are still on the tree?
\item How many total plums are there?
\item How many rotten plums did the farmer collect?
\item How many rotten plums were there in total?
\end{itemize}

\section{The Representation}

\subsection{Observable variables}
\begin{center}
 \begin{tabular}{| c | l |} 
 \hline
 \bf{\em{Symbol}}& \bf{\em{Notes}} \\ [0.5ex] 
 \hline\hline
 t & Time, unit is day  \\ 
 \hline
 P & Total plums  \\
 \hline
 R & Number of rotten plums on the tree  \\
 \hline
 G & Number of good plums  \\
 \hline
\end{tabular}
\end{center}

\subsection{Hidden variable and parameters}
\begin{center}
\begin{tabular}{| c | l |}
\hline
\bf{\em{Symbol}}  & \bf{\em{Notes}} \\
\hline
$r_1$ & The probability of disease infections. In this case, $r_1 = 0.35$ \\
\hline
$f_1$ &
	\begin{tabular}{@{}l@{}}Farmer factor-1,to activate the rotting infection. In particular,\\ 
		\quad $f_1 =
			\left\{
			    \begin{array}{ll}
	        			0  &  \mbox{for day 1 \& 2 after farmer's dis-infection day} \\
	        			1  &  \mbox{for other days} \\
	    		   \end{array}
		\right . $ 
	\end{tabular} \\  
\hline
$f_2$ & 
	\begin{tabular}{@{}l@{}}Farmer factor-2,to remote rotten plums. In particular,\\ 
		\quad $f_2 =
			\left\{
			    \begin{array}{ll}
	        			r_2  & \mbox{on farmer's dis-infection day. In Lewy's case, $r_2=0.75$}\\
	        			0  &  \mbox{else where} 
	    		   \end{array}
		\right . $ 
	\end{tabular} \\  
\hline
M & Number of plums subjected to the rotting diseases, i.e. good plums $>$ 2 days \\
\hline
N & Temporary variable ($N=R + r_1 * M * f_1$), representing the number of diseased plums. \\
\hline
Q & The number of new plums per day. In Lewy's statement, $Q=3$ \\
\hline

\end{tabular}
\end{center}
\end{document}  